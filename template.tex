% Dit is de "preamble" van het document. Hier worden de opmaakopties ingesteld.
% Pro-tip: dingen na het % teken worden niet gecompileerd door LaTeX. Ik zal ze uitgebreid gebruiken om dingen uit te leggen.
% Opmerking: niet om je af te schrikken van LaTeX, maar het is normaal om problemen te hebben. Ik heb de copyright info onderaan het document toegevoegd van de persoon die dit pakket heeft geschreven, omdat zijn documentatie niet helemaal overeenkomt met hoe het daadwerkelijk wordt gebruikt. Dus dit is een combinatie van zijn werkende inleiding samen met mijn toegevoegde commentaar of uitleg. 
% ================================================================

\documentclass[stu,12pt,floatsintext]{apa7}
% Document class input explanation ________________
% LaTeX-bestanden moeten beginnen met de documentklasse, zodat het weet wat het gebruikt.
% - Dit bestand gebruikt de apa7 document class, omdat daar veel opmaak in is ingebouwd.
% Er zijn twee sets haakjes in LaTeX, voor elk commando (de dingen die beginnen met de schuine streep )
% - De tilde haakjes {} zijn verplicht voor het uitvoeren van de opdracht
% - De vierkante haakjes [] zijn opties voor dat commando. Er kan meer dan één set vierkante haakjes zijn voor sommige commando's
% Opties gebruikt in dit document (algemene opmerking - voor elk van deze, als je de andere opties wilt gebruiken, verwissel het dan op die plek in de vierkante haken):
% - stu: this sets the `document mode' as the "student paper" version. Other options are jou (journal), man (manuscript, for journal submission), and doc (a plain document)
% --- De instelling 'student' bevat parameters als 'duedate', 'course' en 'professor' op de titelpagina. Als deze niet gewenst/nodig zijn, gebruik dan de 'man' instelling. De tabellen en figuren worden ook standaard aan het einde van het document toegevoegd. Dit kun je veranderen door de optie 'floatsintext' op te nemen, zoals ik voor jou heb gedaan. Als de docent ze aan het einde wil hebben, verwijder dat dan uit de vierkante haken.
% --- De instelling voor manuscripten is ongeveer wat je zou gebruiken om in te dienen bij een tijdschrift, dus gebruikt 'date' in plaats van 'duedate' en bevat geen informatie over 'course' of 'professor'. Net als bij 'stu' worden tabellen en figuren standaard aan het einde geplaatst in plaats van in de tekst. Dezelfde optie plaatst deze afbeeldingen in de tekst.
% --- Journal ('jou') geeft iets weer dat lijkt op een gewoon tijdschriftformaat - tekst in dubbele kolommen en figuren op hun plaats. Dit kan leuk zijn, vooral als je dit als schrijfvoorbeeld in applicaties gebruikt.
% --- Document ('doc') geeft de tekst weer in één kolom, met één interlinie en figuren op hun plaats. Een alternatieve optie voor het produceren van een meer getypeset uitziend document als schrijfvoorbeeld.
% - 12pt: stelt de lettergrootte in op 12pt. Andere opties zijn 10pt of 11pt
% - floatsintext: zorgt ervoor dat tabellen en figuren in de tekst verschijnen in plaats van aan het einde.

\usepackage[dutch]{babel} % Taalkeuze
\usepackage{hyperref} % Hyperlinks
\usepackage{csquotes} % Noodzakelijk in combinatie met babel.

\usepackage[style=apa,sortcites=true,sorting=nyt,backend=biber]{biblatex} % Refereren in LaTeX volgens APA7.
\DeclareLanguageMapping{dutch}{dutch-apa} % Taalkeuze referentiestijl.
\addbibresource{bibliografie.bib} % Pas dit aan naar de naam van jouw .bib bestand.

\usepackage[T1]{fontenc} % Times New Roman.
\usepackage{mathptmx} % Wiskundefont.

% ================================================================

% Titelpagina _____________________
\title{Grote Lange Versie van de Titel voor de Titelpagina}
\shorttitle{Korte Titel}
\author{[Jouw Naam Hier]}
\duedate{\today}
% \date{January 17, 2024}  % De studentenversie gebruikt het commando \datum niet, om wat voor reden dan ook.
\authorsaffiliations{ASO Spijker}
\course{5 Wetenschappelijke Methodiek}  % Laat dit leeg als je dit veld niet nodig hebt.
\professor{Pieter Smets}  % Laat dit leeg als je dit veld niet nodig hebt.

\abstract{[Hier komt de samenvatting]}

\keywords{kernwoord 1, kernwoord 2, nog eentje} % Als je trefwoorden nodig hebt voor je paper, verwijder dan de % aan het begin van deze regel

\begin{document}
\maketitle  % De standaard in LaTeX voor het aanmaken van de titelpagina.

% \section{Introductie} 

...

\section{Midden}

...

\section{Discussie}

...

\printbibliography

\end{document}

%% 
%% Copyright (C) 2019 by Daniel A. Weiss <daniel.weiss.led at gmail.com>
%% 
%% This work may be distributed and/or modified under the
%% conditions of the LaTeX Project Public License (LPPL), either
%% version 1.3c of this license or (at your option) any later
%% version.  The latest version of this license is in the file:
%% 
%% http://www.latex-project.org/lppl.txt
%% 
%% Users may freely modify these files without permission, as long as the
%% copyright line and this statement are maintained intact.
%% 
%% This work is not endorsed by, affiliated with, or probably even known
%% by, the American Psychological Association.
%% 
%% This work is "maintained" (as per LPPL maintenance status) by
%% Daniel A. Weiss.
%% 
%% This work consists of the file  apa7.dtx
%% and the derived files           apa7.ins,
%%                                 apa7.cls,
%%                                 apa7.pdf,
%%                                 README,
%%                                 APA7american.txt,
%%                                 APA7british.txt,
%%                                 APA7dutch.txt,
%%                                 APA7english.txt,
%%                                 APA7german.txt,
%%                                 APA7ngerman.txt,
%%                                 APA7greek.txt,
%%                                 APA7czech.txt,
%%                                 APA7turkish.txt,
%%                                 APA7endfloat.cfg,
%%                                 Figure1.pdf,
%%                                 shortsample.tex,
%%                                 longsample.tex, and
%%                                 bibliography.bib.
%% 
%%
%%
%% This is file `./samples/shortsample.tex',
%% generated with the docstrip utility.
%%
%% The original source files were:
%%
%% apa7.dtx  (with options: `shortsample')
%% ----------------------------------------------------------------------
%% 
%% apa7 - A LaTeX class for formatting documents in compliance with the
%% American Psychological Association's Publication Manual, 7th edition
%% 
%% Copyright (C) 2019 by Daniel A. Weiss <daniel.weiss.led at gmail.com>
%% 
%% This work may be distributed and/or modified under the
%% conditions of the LaTeX Project Public License (LPPL), either
%% version 1.3c of this license or (at your option) any later
%% version.  The latest version of this license is in the file:
%% 
%% http://www.latex-project.org/lppl.txt
%% 
%% Users may freely modify these files without permission, as long as the
%% copyright line and this statement are maintained intact.
%% 
%% This work is not endorsed by, affiliated with, or probably even known
%% by, the American Psychological Association.
%% 
%% ----------------------------------------------------------------------